\capitulo{4}{Técnicas y herramientas}

\section{Moodle}\label{sec:moodle-local}
\section{Sublime Text}
Para la edición del los diferentes \textit{scripts} utilizaremos este editor de texto ya que, además de ser gratuito (aunque un tanto cargante con solicitar la compra de la versión de pago), es intuitivo y nos proporciona una interfaz cómoda para trabajar,además de una función de auto completar altamente útil
\section{Tex Studio}
\section{JSONMate}
Hemos utilizado esta herramienta, la cual en realidad es una página web que nos ayuda a interpretar la información obtenida en formato \textit{JSON} y simplificarnos su vista~\cite{json:jsnomate}.
