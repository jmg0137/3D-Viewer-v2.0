\capitulo{4}{Técnicas y herramientas}

\section{Moodle}\label{sec:moodle-local}


\section{Sublime Text}
Para la edición del los diferentes \textit{scripts} utilizaremos este editor de texto ya que, además de ser gratuito (aunque un tanto cargante con solicitar la compra de la versión de pago), es intuitivo y nos proporciona una interfaz cómoda para trabajar,además de una función de auto completar altamente útil

\section{Tex Studio}


\section{JSONMate}
Hemos utilizado esta herramienta, la cual en realidad es una página web que nos ayuda a interpretar la información obtenida en formato \textit{JSON} y simplificarnos su vista~\cite{json:jsnomate}.

\section{PuTTY}


\section{Servidor arquimedes}\label{sec:arquimedes}


\section{Comparativa servidores para desplegar Flask}
Dado que la Universidad de Burgos nos ha dotado con un servidor (~\ref{sec:arquimedes}) tenemos que elegir la manera de desplegar nuestra \textit{API Flask}, con lo cual hemos realizado una comparativa con diversas herramientas para desplegarla con el fin de encontrar la mejor manera de hacerlo.

A continuación, mostraremos una tabla con las diferentes herramientas seleccionadas con el fin de elegir la que mejor se amolde a nuestro caso:

\begin{table}
	\centering
	\caption{Tabla comparativa servidores}
	\label{serverTable}
	\begin{tabular}{c c c c c}
		\toprule
		\backslashbox{}{} & \textbf{Apache} & \textbf{uWSGI} & \textbf{Stand-Alone (Gunicorn)} & \textbf{Stand-Alone (Twisted Web)} \\
		\midrule
		\textbf{¿Por qué utilizarlo?} & En el caso de tener experiencia utilizando Apache, así como que se tenga una dependencia del mismo, esto significará \textbf{estabilidad} en el entorno de producción de la aplicación, teniendo gran variedad de módulos estable sy completos. A su vez, es un software muy probado y fiable, teneindo gran variedad de información en la web. & Soporta aplicaciones Python por completo corriendo en WSGI, pudiendo sus componentes realizar muchas más funciones que correr la aplicación, con la correspondiente bajón en el uso de la memoria. Como \textbf{desventaja} hemos considerado que, como está actualmente en desarrollo, podría conllevar a un fallo que aún no se halla contemplado, teniendo a su vez una convención de nombres confusa. & \textbf{Gunicorn}: Si se desea extender de \textit{Apache} utilizando Python(siempre y cuando sea necesario) y programarlo para alguna tarea en concreto. Además, tiene la ventaja de ser sencillo de ejecutar si no se necesita extender de \textit{Apache} & \textbf{Twisted Web}: Si se desea extender de \textit{Apache} utilizando Python(siempre y cuando sea necesario ) siendo simple, estable y maduro. A su vez, puede soportar clientes virtuales.\\
		\bottomrule
	\end{tabular}
\end{table}