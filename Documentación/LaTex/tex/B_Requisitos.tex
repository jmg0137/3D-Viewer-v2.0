\apendice{Especificación de Requisitos}

\section{Introducción}
En este apéndice expondremos los diferentes objetivos que debe lograr la aplicación, así como los requisitos que debe cumplir.

\section{Objetivos generales}
Ahora vamos a listar los objetivos que el proyecto debe alcanzar:
\begin{itemize}
	\item Crear un visor \textit{web} para mostrar modelos tridimensionales previamente digitalizados.
	\item Construir un servidor que proporcione la aplicación \textit{web}.
	\item Detección y corrección de elementos sobrantes en modelos digitales.
\end{itemize}

\section{Catálogo de requisitos}
Vamos a definir los requisitos extraídos a partir de los objetivos generales del proyecto.

\subsection{Requisitos funcionales}
\begin{itemize}
	\item \textbf{RF-1 Gestión de anotaciones}: la aplicación debe ser capaz de manejar anotaciones sobre el modelo.
	\begin{itemize}
		\item \textbf{RF-1.1 Añadir anotación}: el usuario debe poder añadir una anotación.
		\item \textbf{RF-1.2 Eliminar anotación}: el usuario debe poder eliminar una anotación.
		\item \textbf{RF-1.3 Editar anotación}: el usuario debe poder editar la etiqueta de una anotación.
		\item \textbf{RF-1.4 Seleccionar anotación}: el usuario debe poder seleccionar una anotación.
		\item \textbf{RF-1.5 Deseleccionar anotación}: el usuario debe poder deseleccionar una anotación.
	\end{itemize}
	\item \textbf{RF-2 Gestión de medidas}: la aplicación debe ser capaz de manejar medidas sobre el modelo.
	\begin{itemize}
		\item \textbf{RF-2.1 Añadir medida}: el usuario debe poder añadir una medida.
		\item \textbf{RF-2.2 Eliminar medida}: el usuario debe poder eliminar una medida.
		\item \textbf{RF-2.3 Editar medida}: el usuario debe poder editar la etiqueta de una medida.
		\item \textbf{RF-2.4 Seleccionar medida}: el usuario debe poder seleccionar una medida.
		\item \textbf{RF-2-5 Deseleccionar medida}: el usuario debe poder deseleccionar una medida.
	\end{itemize}
	\item \textbf{RF-3 Exportación de puntos}: la aplicación debe ser capaz de persistir las anotaciones que maneje en el momento sobre un modelo.
	\item \textbf{RF-4 Importación de puntos}: la aplicación debe ser capaz de importar anotaciones previamente guardadas.
	\item \textbf{RF-5 Visualizar modelo}: la aplicación debe ser capaz de visualizar un modelo.
	\item \textbf{RF-6 Gestión de modelos}: la aplicación debe ser capaz de mostrar de una pasada los diferentes modelos disponibles.
	\begin{itemize}
		\item \textbf{RF-6.1 Listar modelos}: la aplicación debe ser capaz de listar el conjunto de modelos disponibles.
		\item \textbf{RF-6.2 Añadir modelos}: el usuario debe poder añadir nuevos modelos al catálogo disponible.
	\end{itemize}
\end{itemize}

\subsection{Requisitos no funcionales}
\begin{itemize}
	\item \textbf{RNF-1 Usabilidad}: el conjunto de elementos visuales de la interfaz deben ser intuitivos y conocidos por el usuario medio, permitiendo un rápido aprendizaje.
	\item \textbf{RNF-2 Mantenibilidad}: la aplicación debe desarrollarse siguiendo alguna técnica que permita facilidad de mantenimiento e incorporación de nuevas características, así como corrección de errores.
	\item \textbf{RNF-3 Soporte}: la aplicación debe poder emplearse sobre un amplio conjunto de navegadores.
	\item \textbf{RNF-4 Internacionalización}: la aplicación debe estar diseñada para soportar diferentes idiomas.
	\item \textbf{RNF-5 Control de acceso}: la aplicación debe soportar control de acceso de usuarios.
\end{itemize}

\section{Especificación de requisitos}

\subsection{Actores}
En nuestro caso solamente tendremos dos actores, que serán el alumno y el administrador.

\subsection{Diagrama de casos de uso}\label{diagrama-de-casos-de-uso}
Podemos ver en~\ref{fig:use-case-general-diagram} el diagrama de casos de uso completo.

\begin{landscape}
	\imagenpersonalizada{use-case-general-diagram}{Diagrama de casos de uso general}{1.35}
\end{landscape}


A partir de los diagramas presentados, hemos creado los casos de uso del visor de modelos~\ref{tabla:use-case-1}, del listado de los modelos~\ref{tabla:use-case-2} y de la subida de modelos~\ref{tabla:use-case-3}.

En el caso de uso del visor de modelos, éste se divide en la gestión de anotaciones~\ref{tabla:use-case-1.1}, en la gestión de medidas~\ref{tabla:use-case-1.2}, en exportar puntos~\ref{tabla:use-case-1.3} y en importar puntos~\ref{tabla:use-case-1.4}.

Siguiendo por el gestor de anotaciones~\ref{tabla:use-case-1.1}, tenemos añadir anotación~\ref{tabla:use-case-1.1.1}, eliminar anotación~\ref{tabla:use-case-1.1.2}, editar anotación~\ref{tabla:use-case-1.1.3}, seleccionar anotación~\ref{tabla:use-case-1.1.4} y deseleccionar anotación~\ref{tabla:use-case-1.1.5}.

Finalmente, para la gestión de medidas~\ref{tabla:use-case-1.2} tenemos añadir medida~\ref{tabla:use-case-1.2.1}, eliminar medida~\ref{tabla:use-case-1.2.2}, editar medida~\ref{tabla:use-case-1.2.3}, seleccionar medida~\ref{tabla:use-case-1.2.4} y deseleccionar medida~\ref{tabla:use-case-1.2.5}.



\tablaAncho
{CU-01 Visor de modelos}
{p{2.9cm} X}
{use-case-1}
{	
	\textbf{CU-01} & \textbf{Visor de modelo} \\ \otoprule
	\textbf{Versión} & 1.0 \\ \midrule
	\textbf{Autor} & Alberto Vivar Arribas \\ \midrule
	\textbf{Requisitos asociados} & RF-1, RF-1.1, RF-1.2, RF-1.3, RF-1.4, RF-1.5, RF-2, RF-2.1, RF-2.2, RF-2.3, RF-2.4, RF-2.5, RF-3, RF-4, RF-5 \\ \midrule
	\textbf{Descripción} & Permite al usuario realizar aplicaciones sobre un modelo concreto. \\ \midrule
	\textbf{Precondiciones} & 
		\tabitem El usuario ha seleccionado un modelo.
		\\ \midrule
	\textbf{Acciones} & 
		\enumeratecompacto{
			\item El usuario abre un modelo.
			\item Se muestran el modelo sin anotaciones ni medidas.
			\item Se da la posibilidad de gestionar tanto anotaciones como modelos.
		}
	 \\ \midrule
	\textbf{Postcondiciones} & - \\ \midrule
	\textbf{Excepciones} & - \\ \midrule
	\textbf{Importancia} & Alta \\ 
}


\tablaAncho
{CU-1.1 Gestión de anotaciones}
{p{2.9cm} X}
{use-case-1.1}
{
	\textbf{CU-1.1} & \textbf{Gestión de anotaciones} \\ \otoprule
	\textbf{Versión} & 1.0 \\ \midrule
	\textbf{Autor} & Alberto Vivar Arribas \\ \midrule
	\textbf{Requisitos asociados} & RF-1.1, RF-1.1.1, RF-1.1.2, RF-1.1.3, RF-1.1.4, RF-1.1.5 \\ \midrule
	\textbf{Descripción} & Permite gestionar las anotaciones (añadirlas, eliminarlas, etc.). \\ \midrule
	\textbf{Precondiciones} & 
		\tabitem El usuario tiene abierto un modelo.
		\\ \midrule
	\textbf{Acciones} & 
	\enumeratecompacto{
		\item Se muestran las anotaciones del modelo visualizado.
		\item Se muestran las opciones de añadir, eliminar, editar la etiqueta de y deseleccionar las anotaciones.
	}
	\\ \midrule
	\textbf{Postcondiciones} & - \\ \midrule
	\textbf{Excepciones} & - \\ \midrule
	\textbf{Importancia} & Alta \\ 
}


\tablaAncho
{CU-1.1.1 Añadir anotación}
{p{2.9cm} X}
{use-case-1.1.1}
{	
	\textbf{CU-1.1.1} & \textbf{Añadir anotación} \\ \otoprule
	\textbf{Versión} & 1.0 \\ \midrule
	\textbf{Autor} & Alberto Vivar Arribas \\ \midrule
	\textbf{Requisitos asociados} & RF-1.1 \\ \midrule
	\textbf{Descripción} & Permite al usuario añadir una anotación al modelo. \\ \midrule
	\textbf{Precondiciones} & - \\ \midrule
	\textbf{Acciones} & 
	\enumeratecompacto{
		\item El usuario pincha en añadir anotación.
		\item El usuario pincha sobre el modelo en la zona donde quiere crear una anotación.
		\item Una esfera es añadida en el modelo para representar la anotación.
		\item Un elemento aparece en un menú para representar la etiqueta de la anotación.
	}
	\\ \midrule
	\textbf{Postcondiciones} & 
		\tabitem Se añade una anotación al modelo.
		\\ \midrule
	\textbf{Excepciones} & - \\ \midrule
	\textbf{Importancia} & Alta \\ 
}


\tablaAncho
{CU-1.1.2 Eliminar anotación}
{p{2.9cm} X}
{use-case-1.1.2}
{	
	\textbf{CU-1.1.2} & \textbf{Eliminar anotación} \\ \otoprule
	\textbf{Versión} & 1.0 \\ \midrule
	\textbf{Autor} & Alberto Vivar Arribas \\ \midrule
	\textbf{Requisitos asociados} & RF-1.2 \\ \midrule
	\textbf{Descripción} & Permite al usuario eliminar una anotación. \\ \midrule
	\textbf{Precondiciones} & 
		\tabitem Que exista una anotación en el modelo.
		
		\tabitem (opcional) Que una anotación se encuentre seleccionada.
		\\ \midrule
	\textbf{Acciones} & 
	\enumeratecompacto{
		\item Seleccionar una anotación (opcional).
		\item Pulsar sobre borrar anotación. Si se ha realizado anterior, saltamos el siguiente paso.
		\item Seleccionar una anotación (opcional).
		\item Se borra la anotación.
	}
	\\ \midrule
	\textbf{Postcondiciones} & 
		\tabitem Se elimina la anotación seleccionada.
		\\ \midrule
	\textbf{Excepciones} & - \\ \midrule
	\textbf{Importancia} & Media \\ 
}


\tablaAncho
{CU-1.1.3 Editar anotación}
{p{2.9cm} X}
{use-case-1.1.3}
{	
	\textbf{CU-1.1.3} & \textbf{Editar anotación} \\ \otoprule
	\textbf{Versión} & 1.0 \\ \midrule
	\textbf{Autor} & Alberto Vivar Arribas \\ \midrule
	\textbf{Requisitos asociados} & RF-1.3 \\ \midrule
	\textbf{Descripción} & Permite editar la etiqueta de una anotación. \\ \midrule
	\textbf{Precondiciones} & 
		\tabitem Que haya anotaciones.
		\\ \midrule
	\textbf{Acciones} & 
	\enumeratecompacto{
		\item Seleccionar una anotación.
		\item Pulsar sobre editar anotación.
		\item Se muestra un diálogo con la etiqueta actual de la anotación.
		\item El usuario edita el texto.
		\item Pulsar sobre guardar.
		\item El texto de la anotación cambia.
	}
	\\ \midrule
	\textbf{Postcondiciones} & 
		\tabitem La anotación modificada cambia su texto.
		\\ \midrule
	\textbf{Excepciones} & - \\ \midrule
	\textbf{Importancia} & Media \\ 
}


\tablaAncho
{CU-1.1.4 Seleccionar anotación}
{p{2.9cm} X}
{use-case-1.1.4}
{
	\textbf{CU-1.1.4} & \textbf{Seleccionar anotación} \\ \otoprule
	\textbf{Versión} & 1.0 \\ \midrule
	\textbf{Autor} & Alberto Vivar Arribas \\ \midrule
	\textbf{Requisitos asociados} & RF-1.4 \\ \midrule
	\textbf{Descripción} & Permite seleccionar una anotación resaltándola. \\ \midrule
	\textbf{Precondiciones} & 
		\tabitem Tiene que existir al menos una anotación.
		\\ \midrule
	\textbf{Acciones} & 
	\enumeratecompacto{
		\item El usuario selecciona una anotación en el modelo (opción 1).
		\item El usuario selecciona una anotación en la lista lateral (opción 2).
		\item La anotación se destaca tanto en el modelo como en la lista lateral.
	}
	\\ \midrule
	\textbf{Postcondiciones} & - \\ \midrule
	\textbf{Excepciones} & - \\ \midrule	
	\textbf{Importancia} & Media \\ 
}


\tablaAncho
{CU-1.1.5 Deseleccionar anotación}
{p{2.9cm} X}
{use-case-1.1.5}
{
	\textbf{CU-1.1.5} & \textbf{Deseleccionar anotación} \\ \otoprule
	\textbf{Versión} & 1.0 \\ \midrule
	\textbf{Autor} & Alberto Vivar Arribas \\ \midrule
	\textbf{Requisitos asociados} & RF-1.5 \\ \midrule
	\textbf{Descripción} & Permite al usuario deseleccionar una anotación. \\ \midrule
	\textbf{Precondiciones} & 
		\tabitem Que haya al menos una anotación seleccionada.
		\\ \midrule
	\textbf{Acciones} & 
	\enumeratecompacto{
		\item Pinchar sobre una anotación en el modelo (opción 1).
		\item Pinchar sobre una anotación en la lista lateral (opción 2).
		\item Pinchar sobre deseleccionar todo en la lista lateral (opción 3).
		\item Si se ha realizado opcion 1 o 2, se deselecciona dicha anotación.
		\item Si se realiza opción 3, se deseleccionan todas las anotaciones.
	}
	\\ \midrule
	\textbf{Postcondiciones} & - \\ \midrule
	\textbf{Excepciones} & - \\ \midrule
	\textbf{Importancia} & Media \\ 
}


\tablaAncho
{CU-1.2 Gestión de medidas}
{p{2.9cm} X}
{use-case-1.2}
{
	\textbf{CU-1.2} & \textbf{Gestión de medida} \\ \otoprule
	\textbf{Versión} & 1.0 \\ \midrule
	\textbf{Autor} & Alberto Vivar Arribas \\ \midrule
	\textbf{Requisitos asociados} & RF-1.2, RF-1.2.1, RF-1.2.2, RF-1.2.3, RF-1.2.4, RF-1.2.5 \\ \midrule
	\textbf{Descripción} & Permite gestionar las medidas (añadirlas, eliminarlas, \dots). \\ \midrule
	\textbf{Precondiciones} & 
		\tabitem El usuario tiene abierto un modelo.
		\\ \midrule
	\textbf{Acciones} & 
	\enumeratecompacto{
		\item Se muestran las medidas del modelo visualizado.
		\item Se muestran las opciones de añadir, eliminar, editar la etiqueta de y deseleccionar las medidas.
	}
	\\ \midrule
	\textbf{Postcondiciones} & - \\ \midrule
	\textbf{Excepciones} & - \\ \midrule
	\textbf{Importancia} & Alta \\ 
}


\tablaAncho
{CU-1.2.1 Añadir medida}
{p{2.9cm} X}
{use-case-1.2.1}
{
	\textbf{CU-1.2.1} & \textbf{Añadir medida} \\ \otoprule
	\textbf{Versión} & 1.0 \\ \midrule
	\textbf{Autor} & Alberto Vivar Arribas \\ \midrule
	\textbf{Requisitos asociados} & RF-2.1 \\ \midrule
	\textbf{Descripción} & Permite al usuario añadir una medida al modelo. \\ \midrule
	\textbf{Precondiciones} & - \\ \midrule
	\textbf{Acciones} & 
	\enumeratecompacto{
		\item El usuario pincha en añadir medida.
		\item El usuario pincha sobre el modelo en la zona donde quiere crear una anotación.
		\item El usuario pincha sobre el modelo en la posición donde quiere que vaya el otro extremo de la medida.
		\item Dos esferas y una raya aparecen en el visor para representar la medida.
		\item Un elemento aparece en un menú para representar la etiqueta de la medida, que será la etiqueta y las unidades.
	}
	\\ \midrule
	\textbf{Postcondiciones} & 
		\tabitem Se añade una medida al modelo.
		\\ \midrule
	\textbf{Excepciones} & 
		\tabitem No se añaden los dos puntos de la medida.
		\\ \midrule
	\textbf{Importancia} & Alta \\ 
}


\tablaAncho
{CU-1.2.2 Eliminar medida}
{p{2.9cm} X}
{use-case-1.2.2}
{
	\textbf{CU-1.2.2} & \textbf{Eliminar medida} \\ \otoprule
	\textbf{Versión} & 1.0 \\ \midrule
	\textbf{Autor} & Alberto Vivar Arribas \\ \midrule
	\textbf{Requisitos asociados} & RF-2.2 \\ \midrule
	\textbf{Descripción} & Permite al usuario eliminar una medida. \\ \midrule
	\textbf{Precondiciones} & 
		\tabitem Que exista una medida en el modelo.
		
		\tabitem (opcional) Que una medida se encuentre seleccionada.
		\\ \midrule
	\textbf{Acciones} & 
	\enumeratecompacto{
		\item Seleccionar una medida (opción 1).
		\item Pulsar sobre borrar medida.
		\item Seleccionar una medida (opción 2).
		\item Se borra la medida.
	}
	\\ \midrule
	\textbf{Postcondiciones} & 
		\tabitem Se elimina la medida seleccionada.
		\\ \midrule
	\textbf{Excepciones} & - \\ \midrule
	\textbf{Importancia} & Media \\ 
}


\tablaAncho
{CU-1.2.3 Editar medida}
{p{2.9cm} X}
{use-case-1.2.3}
{
	\textbf{CU-1.2.3} & \textbf{Editar medida} \\ \otoprule
	\textbf{Versión} & 1.0 \\ \midrule
	\textbf{Autor} & Alberto Vivar Arribas \\ \midrule
	\textbf{Requisitos asociados} & RF-2.3 \\ \midrule
	\textbf{Descripción} & Permite editar la etiqueta de una medida. \\ \midrule
	\textbf{Precondiciones} & 
		\tabitem Que haya medidas.
		\\ \midrule
	\textbf{Acciones} & 
	\enumeratecompacto{
		\item Seleccionar una medida.
		\item Pulsar sobre editar medida.
		\item Mostrar un diálogo con la etiqueta actual de la medida.
		\item El usuario edita el texto.
		\item Pulsar sobre guardar.
		\item El texto de la medida cambia.
	}
	\\ \midrule
	\textbf{Postcondiciones} & 
		\tabitem La medida modificada cambia su texto.
		\\ \midrule
	\textbf{Excepciones} & - \\ \midrule
	\textbf{Importancia} & Media \\ 
}


\tablaAncho
{CU-1.2.4 Seleccionar medida}
{p{2.9cm} X}
{use-case-1.2.4}
{
	\textbf{CU-1.2.4} & \textbf{Seleccionar medida} \\ \otoprule
	\textbf{Versión} & 1.0 \\ \midrule
	\textbf{Autor} & Alberto Vivar Arribas \\ \midrule
	\textbf{Requisitos asociados} & RF-2.4 \\ \midrule
	\textbf{Descripción} & Permite seleccionar una medida resaltándola. \\ \midrule
	\textbf{Precondiciones} & 
		\tabitem Tiene que existir al menos una medida.
		\\ \midrule
	\textbf{Acciones} & 
	\enumeratecompacto{
		\item El usuario selecciona una medida en el modelo (opción 1).
		\item El usuario selecciona una medida en la lista lateral (opción 2).
		\item La medida se destaca tanto en el modelo como en la lista lateral.
	}
	\\ \midrule
	\textbf{Postcondiciones} & - \\ \midrule
	\textbf{Excepciones} & - \\ \midrule
	\textbf{Importancia} & Media \\ 
}


\tablaAncho
{CU-1.2.5 Deseleccionar medida}
{p{2.9cm} X}
{use-case-1.2.5}
{
	\textbf{CU-1.2.5} & \textbf{Deseleccionar medida} \\ \otoprule
	\textbf{Versión} & 1.0 \\ \midrule
	\textbf{Autor} & Alberto Vivar Arribas \\ \midrule
	\textbf{Requisitos asociados} & RF-2.5 \\ \midrule
	\textbf{Descripción} & Permite al usuario deseleccionar una medida. \\ \midrule
	\textbf{Precondiciones} & 
		\tabitem Que haya al menos una medida seleccionada.
		\\ \midrule
	\textbf{Acciones} & 
	\enumeratecompacto{
		\item Pinchar sobre una medida en el modelo (opción 1).
		\item Pinchar sobre una medida en la lista lateral (opción 2).
		\item Pinchar sobre deseleccionar todo en la lista lateral (opción 3).
		\item Si se ha realizado opcion 1 o 2, se deselecciona dicha medida.
		\item Si se realiza opción 3, se deseleccionan todas las medidas.
	}
	\\ \midrule
	\textbf{Postcondiciones} & - \\ \midrule
	\textbf{Excepciones} & - \\ \midrule
	\textbf{Importancia} & Media \\ 
}


\tablaAncho
{CU-1.3 Exportar puntos}
{p{2.9cm} X}
{use-case-1.3}
{
	\textbf{CU-1.3} & \textbf{Exportar punto} \\ \otoprule
	\textbf{Versión} & 1.0 \\ \midrule
	\textbf{Autor} & Alberto Vivar Arribas \\ \midrule
	\textbf{Requisitos asociados} & RF-3 \\ \midrule
	\textbf{Descripción} & Permite exportar los puntos que se estén visualizando. \\ \midrule
	\textbf{Precondiciones} & - \\ \midrule
	\textbf{Acciones} & 
	\enumeratecompacto{
		\item Pulsar sobre exportar puntos.
		\item Seleccionar un directorio de descarga.
		\item Seleccionar un nombre para el archivo de destino.
		\item Pulsar sobre guardar.
		\item Un archivo se descarga.
	}
	\\ \midrule
	\textbf{Postcondiciones} & 
		\tabitem Un archivo nuevo con nuestros puntos se almacena.
		\\ \midrule
	\textbf{Excepciones} & - \\ \midrule
	\textbf{Importancia} & Baja \\ 
}


\tablaAncho
{CU-1.4 Importar puntos}
{p{2.9cm} X}
{use-case-1.4}
{
	\textbf{CU-1.4} & \textbf{Importar punto} \\ \otoprule
	\textbf{Versión} & 1.0 \\ \midrule
	\textbf{Autor} & Alberto Vivar Arribas \\ \midrule
	\textbf{Requisitos asociados} & RF.4 \\ \midrule
	\textbf{Descripción} & Permite al usuario cargar puntos creados previamente. \\ \midrule
	\textbf{Precondiciones} & 
		\tabitem Tener un archivo correctamente formado con puntos para el modelo.
		\\ \midrule
	\textbf{Acciones} & 
	\enumeratecompacto{
		\item Pulsar sobre importar puntos.
		\item Seleccionar el archivo deseado.
		\item Pulsar sobre abrir.
		\item Las anotaciones y medidas se añaden.
	}
	\\ \midrule
	\textbf{Postcondiciones} & 
		\tabitem Se añaden añaden nuevas anotaciones y medidas.
		\\ \midrule
	\textbf{Excepciones} &
		\tabitem El archivo está mal formado.
		
		\tabitem El archivo no contiene anotaciones y medidas.
		\\ \midrule	
	\textbf{Importancia} & Baja \\ 
}


\tablaAncho
{CU-2 Listado de modelos}
{p{2.9cm} X}
{use-case-2}
{
	\textbf{CU-2} & \textbf{Listado de modelo} \\ \otoprule
	\textbf{Versión} & 1.0 \\ \midrule
	\textbf{Autor} & Alberto Vivar Arribas \\ \midrule
	\textbf{Requisitos asociados} & RF-6.1 \\ \midrule
	\textbf{Descripción} & Permite visualizar de un vistazo el conjunto de modelos disponibles. \\ \midrule
	\textbf{Precondiciones} & 
		\tabitem Que haya modelos disponibles almacenados en el servidor.
		\\ \midrule
	\textbf{Acciones} & 
	\enumeratecompacto{
		\item El usuario entra en el catálogo de modelos.
		\item Se muestra el conjunto de modelos disponibles.
	}
	\\ \midrule
	\textbf{Postcondiciones} & - \\ \midrule
	\textbf{Excepciones} & - \\ \midrule
	\textbf{Importancia} & Media-Alta \\ 
}


\tablaAncho
{CU-3 Subida modelos}
{p{2.9cm} X}
{use-case-3}
{
	\textbf{CU-3} & \textbf{Subida de modelo} \\ \otoprule
	\textbf{Versión} & 1.0 \\ \midrule
	\textbf{Autor} & Alberto Vivar Arribas \\ \midrule
	\textbf{Requisitos asociados} & RF-6.2 \\ \midrule
	\textbf{Descripción} & Permite que se añadan nuevos modelos al catálogo. \\ \midrule
	\textbf{Precondiciones} & - \\ \midrule
	\textbf{Acciones} & 
	\enumeratecompacto{
		\item El usuario entra en la sección de subida.
		\item El usuario selecciona un archivo para subir.
		\item El usuario pulsa sobre subir.
		\item El nuevo modelo se almacena en el servidor.
		\item Se muestra un diálogo para confirmar.
	}
	\\ \midrule
	\textbf{Postcondiciones} & 
		\tabitem Existe un modelo más en el catálogo.
		\\ \midrule
	\textbf{Excepciones} &
		\tabitem La descarga es interrumpida.
		\\ \midrule
	\textbf{Importancia} & Media-Alta \\ 
}