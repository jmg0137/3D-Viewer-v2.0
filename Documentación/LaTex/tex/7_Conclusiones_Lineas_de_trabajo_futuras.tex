\capitulo{7}{Conclusiones y Líneas de trabajo futuras}

En esta parte expondremos las conclusiones extraídas a lo largo del trabajo, así como posibles líneas de trabajo futuro con las que continuar el proyecto.

\section{Conclusiones}
Con la ejecución del proyecto, hemos podido extraer una serie de conclusiones:
\begin{itemize}
	\item Hemos conseguido cumplir los objetivos iniciales del proyecto, además del añadido para realizar el análisis de los modelos. Ver sección~\ref{sec:analisis-modelos}.
	\item Que gracias a los conocimientos adquiridos durante la titulación, hemos creado una estructura para el código que facilitará los cambios que se realicen en un futuro.
	\item El empleo de tecnologías y herramientas de los cuales no teníamos experiencia ha retrasado el progreso, pero serán de utilidad a futuro (en especial \LaTeX).
	\item Precisamente el desconocimiento del área sobre el que versa el trabajo ha hecho bastante complicada la estimación del tiempo de las tareas, aunque el uso de metodologías ágiles nos ha aportado la flexibilidad suficiente para abordar dichos imprevistos.
\end{itemize}



\section{Lineas de trabajo futuras}
Ahora pasaremos a explicar algunos aspectos en los que el proyecto puede evolucionar en un futuro. Algunos de ellos se ven reflejados en el apartado \texttt{<<Issues>>} de GitHub.

\subsection{Despliegue en servidor de la UBU}
Sería una buena idea integrar de forma permanente el servicio en un servidor de la universidad por varios motivos. El primero sería la capacidad de mantenerlo como cualquier otro servicio propio de la universidad. Otro sería la seguridad de los datos: al ser los modelos valiosos para la institución, la mejor manera para mantener la privacidad de los mismos es mediante el alojamiento propio.

\subsection{Integración con Moodle}
Otro nivel de integración sería con los servicios ofrecidos actualmente en UBUVirtual. Uno de ellos sería el sistema de cuestionarios que posee. De esta manera, los usuarios podrán realizar los ejercicios en un menor número de pasos y concentrarse mejor en el ejercicio en sí mismo.

\subsection{Perfiles de usuario}
En el momento presente, no existe ningún tipo de perfil de usuario: todos tienen los mismos permisos. Por ello, será necesario implementar algún tipo de control a dicho nivel, para que los profesores o aquellos que administren el contenido sean capaces de limitar y conceder acceso a diferentes elementos de la aplicación.

\subsection{Comercialización}
Finalmente podemos mencionar la posibilidad de monetizar la aplicación, teniendo además varias posibilidades.

Una de ellas sería la de ofrecer acceso a los modelos a otros investigadores o profesores en otras instituciones, probablemente mediante un modelo de suscripción. De esta manera se podría facilitar el trabajo colaborativo sobre el conjunto de modelos.

Y la otra idea es la de desplegar nuestro proyecto sobre otras instituciones con similares necesidades pudiendo añadir además nuevas características, como el soporte de nuevos formatos de archivo, visualizar elementos diferentes como moléculas o nanomateriales\footnote{\url{http://iesbinef.educa.aragon.es/fiqui/jmol/organica.htm?_USE=HTML5}}, etc.