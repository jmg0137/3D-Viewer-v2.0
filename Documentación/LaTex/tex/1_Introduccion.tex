\capitulo{1}{Introducción}

La guía docente del Grado en Historia y Patrimonio contiene algunas asignaturas relacionadas con la antropología, ciencia que estudia las poblaciones humanas, su evolución y el impacto de las mismas.

En la asignatura de Antropología Humana se estudia cómo estimar los diferentes parámetros de un individuo (como el sexo, la edad, altura, etc.) a partir de sus restos óseos. En esta asignatura los laboratorios de prácticas tienen mucha importancia porque se explica presencialmente, dando indicaciones en persona sobre las posiciones de los diferentes elementos con huesos reales o mediante fotografía, explicando sobre la misma las partes a ilustrar.

Una de las grandes dificultades de adaptar un grado de la modalidad presencial a \textit{on-line}\footnote{Los términos docencia \textit{on-line}, virtual y en línea se utilizan indistintamente durante toda la presente memoria.} es la porción impartida en los laboratorios de prácticas. En el primer caso (presencialmente) se requiere físicamente el material y llevarlo hasta el lugar donde se imparta la clase. Esto supone el riesgo de dañar los elementos, en ocasiones inasumible por lo único de la pieza. En el caso de fotografía, se pierde la percepción de profundidad, que puede conllevar una mayor dificultad en el aprendizaje. Por si todo esto no fuera suficiente, sucede que además del Grado en Historia y Patrimonio, también existe su versión on-line, siendo este último el mayor perjudicado por no tener unos mejores medios docentes. Dicho todo esto, es evidente la necesidad de nuevas herramientas que solucionen parte de estos problemas.

Este proyecto busca solventar todos estos inconvenientes, y qué mejor manera que ayudando con los recursos que nos provee la informática. Así, a lo largo de esta memoria vamos a definir nuestra aproximación para un caso real de un Visor 3D para modelos de huesos y fósiles enfocado a la docencia del Grado en Historia y Patrimonio de la UBU.
