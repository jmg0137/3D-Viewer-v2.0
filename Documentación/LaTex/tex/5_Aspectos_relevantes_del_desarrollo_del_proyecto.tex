\capitulo{5}{Aspectos relevantes del desarrollo del proyecto}

Este apartado pretende recoger los aspectos más interesantes del desarrollo del proyecto, comentados por los autores del mismo.
Debe incluir desde la exposición del ciclo de vida utilizado, hasta los detalles de mayor relevancia de las fases de análisis, diseño e implementación.
Se busca que no sea una mera operación de copiar y pegar diagramas y extractos del código fuente, sino que realmente se justifiquen los caminos de solución que se han tomado, especialmente aquellos que no sean triviales.
Puede ser el lugar más adecuado para documentar los aspectos más interesantes del diseño y de la implementación, con un mayor hincapié en aspectos tales como el tipo de arquitectura elegido, los índices de las tablas de la base de datos, normalización y desnormalización, distribución en ficheros3, reglas de negocio dentro de las bases de datos (EDVHV GH GDWRV DFWLYDV), aspectos de desarrollo relacionados con el WWW...
Este apartado, debe convertirse en el resumen de la experiencia práctica del proyecto, y por sí mismo justifica que la memoria se convierta en un documento útil, fuente de referencia para los autores, los tutores y futuros alumnos.

\section{Tratamiento de los roles de los usuarios}
Durante el progreso en la aplicación de partida, nos dimos cuenta de que el tratamiento de los roles podía también ser realizado mediante la cotejación del mismo contra la \textit{API} de UBUVirtual en lugar de tener los roles almacenados en la base de datos junto con los usuarios autorizados. Por otro lado, cabe mencionar que en la aplicación de partida no se estableció ningún tratamiento de roles de usuario, aunque se mencionara.

Inicialmente se decidió llevar a cabo el objetivo inicial de tratamiento de roles, es decir, mediante la definición de los mismos en la base de datos con su posterior consulta a la hora de definir el usuario, aunque después nos dimos cuenta de que la \textit{API} de UBUVirtual podría proporcionarnos estos roles, los cuales vienen dados a cada usuario en la asignatura correspondiente. Para ellos utilizamos las funciones proporcionadas por \textit{Moodle} para realizar peticiones a nuestra \textit{API Rest} (sección \ref{sec:API-Rest}), que en este caso es UBUVirtual. En dicho listado de funciones (~\cite{moodle:web-service-api-functions}) encontramos la función \textit{core_enrol_get_enrolled_users}, la cual nos permitirá conocer los usuarios de la asignatura, así como su rol en la misma y más información variada de cada uno de los participantes. Dicha función nos muestra esta información en forma de diccionario \textit{JSON} desde el que buscaremos al usuario correspondiente para así conocer su rol en la asignatura correspondiente. Dicha información nos es presentada con la siguiente estructura:~\ref{fig:user-info-JSON}

Como se puede apreciar en el campo \textit{roles} nos encontramos con el rol correspondiente del usuario en cuestión, que en este caso es \textit{Profesor} y el id de dicho rol es \textit{3}

\section{Obtención de los Modelos}
La idea inicial de la aplicación era que los modelos proporcionados para su posterior visualización se administraran de manera local, es decir, en una carpeta con todos los modelos. Llegados al punto de pensar cómo podíamos proporcionar al usuario los modelos óseos privados, decidimos que la mejor manera de hacerlo era mediante la administración de dichos modelos como recursos en la \textit{API} de UBUVirtual, siendo estos recursos invisibles para el alumno y a los que solo el profesor tenga acceso para modificar. Para ello, recurrimos de nuevo a las funciones \textit{API Rest} siendo esta vez la función \textit{mod_resource_get_resources_by_courses} la elegida (~\cite{moodle:web-service-api-functions}). Dicha función nos ofrece la información de los recursos presentes en la \textit{API} de UBUVirtual en los cursos correspondientes. En el caso de no seleccionar un curso en concreto, nos devuelve cada uno de los recursos a los que dicho usuario puede acceder. La información resultante tiene la siguiente estructura:~\ref{fig:JSON-resources}