\capitulo{5}{Aspectos relevantes del desarrollo del proyecto}

\section{Tratamiento de los roles de los usuarios}\label{sec:roles}
Durante el progreso en la aplicación de partida, nos dimos cuenta de que el tratamiento de los roles podía también ser realizado mediante la cotejación del mismo contra la \textit{API} de UBUVirtual en lugar de tener los roles almacenados en la base de datos junto con los usuarios autorizados. Por otro lado, cabe mencionar que en la aplicación de partida no se estableció ningún tratamiento de roles de usuario, aunque se mencionara.

Inicialmente se decidió llevar a cabo el objetivo inicial de tratamiento de roles, es decir, mediante la definición de los mismos en la base de datos con su posterior consulta a la hora de definir el usuario, aunque después nos dimos cuenta de que la \textit{API} de UBUVirtual podría proporcionarnos estos roles, los cuales vienen dados a cada usuario en la asignatura correspondiente.

\section{Obtención de los Modelos}
La idea inicial de la aplicación era que los modelos proporcionados para su posterior visualización se administraran de manera local, es decir, en una carpeta con todos los modelos. Llegados al punto de pensar cómo podíamos proporcionar al usuario los modelos óseos privados, decidimos que la mejor manera de hacerlo era mediante la administración de dichos modelos como recursos en la \textit{API} de UBUVirtual, siendo estos recursos invisibles para el alumno y a los que solo el profesor tenga acceso para modificar.

\section{Instalación y configuración de Moodle como API Rest}
Para poder realizar las pruebas pertinentes en cada parte del proyecto, hemos decidido que lo ideal es tener instalado \textit{Moodle} de manera local para realizar las llamadas, así como la subida de recursos, asignación de roles, etc, sin tener que depender de un tutor (el cual puede crear una asignatura ficticia y realizar las pruebas ahí). Por ello, hemos realizado la instalación de \textit{Moodle} con un paquete instalador (sección ~\ref{sec:moodle-local}) para \textit{Windows} en el cual viene incluido \textit{XAMPP}~\cite{wiki:xampp} así como \textit{MySql}~\cite{wiki:mysql}.

Una vez instalado \textit{Moodle} y creado un curso y un alumno para poder realizar las pruebas, nos hemos encontrado con el problema de que la \textit{API} no era una \textit{API Rest}~\ref{sec:api-rest} ya que a la hora de realizar las peticiones necesarias para la obtención de información del usuario se nos denegaba el acceso. Para poder solucionar este problema hemos tenido que configurar nuestra \textit{API} cambiando los parámetros correspondientes (véase \textit{Manual del Programador}).

\section{Encriptado y desencriptado de los modelos}
Con la finalidad de obtener seguridad en nuestra \textit{API} en lo relacionado a los modelos decidimos realizar una operación de encriptado y desencriptado de los mismos para que alguien ajeno a la \textit{API} no sea capaz de obtener los modelos o modificarlos.

Para realizar el encriptado realizamos la lectura del modelo y modificamos ciertos valores con el fin de que a la hora de cargar dicho modelo, el cargador de modelos sea capaz de desencriptar el modelo en cuestión y así visualizarlo. Sin embargo nos hemos encontrado con ciertos problemas a la hora de encriptar y desencriptar los modelos (véase \textit{Manual del Programador}).